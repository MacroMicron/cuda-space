\newpage
\begin{thebibliography}{99}

  \bibitem{sazonov}
  \textit{С.Б.\,Медведев, В.В.\,Сазонов, Х.У.\,Сайгираев}, Моделирование зон неустойчивой работы радиотехнической измерительной системы с активным ответом во время сближения и стыковки космических кораблей с Международной Космической Станцией
  // Математическое моделирование, 24(2):151–160, 2012.
  [\href{http://istina.imec.msu.ru/publications/article/533721/}{html}]

  \bibitem{soyz}
  Легедарный корабль <<Союз>>
  // Новости Космонавтики, апрель 2002


  \bibitem{boreskov1}
  \textit{А.В.\,Боресков, А.А.\,Харламов}, Основы работы с технологией CUDA
  // ДМК Пресс, 2010

  \bibitem{berezin}
  \textit{C.Б.\,Березин, В.М.\,Пасконов, Н.А.\,Сахарных}, Моделирование трехмерных течений методом расщепления с использованием параллельной архитектуры ГПУ
 // Вычислительные методы и программирование, 13: 75-81, 2012.
  [\href{http://num-meth.srcc.msu.ru/zhurnal/tom_2012/pdf/v13r210.pdf}{pdf}]

  \bibitem{cynami}
  \textit{М.А.\,Курако}, Оптимизация производительности вычислений для моделирования цунами на параллельных архитекутурах
  // Молодёжь и наука: Сборник материалов VII Всероссийской научно-технической конференции студентов, аспирантов и молодых учёных, посвященной 50-летию первого полета человека в космос
  [\href{http://elib.sfu-kras.ru/bitstream/2311/5869/1/s3_029.pdf}{pdf}]
  
  \bibitem{neuron}
  \textit{В.В.\,Парубец, О.Г.\,Берестнева, Д.В.\,Девятых}, Применение технологии CUDA для ускорения вычислений в нейронных сетях
  // Известия Томского политехнического университета, том 320, выпуск 5 (2012)

  \bibitem{radiolocation}
  \textit{П.В.\,Маковецкий, В.Г.\,Васильев}, Отражение радиолокационных сигналов -- лекции
  // Ленинградский институт авиационного приборостроения, 1975

  \bibitem{gugens}
  Принцип Гюйгенса-Френеля
  // Большая советская энциклопедия (3-е издание)

  \bibitem{ferma}
  \textit{Schuster A.}, An Introduction to the Theory of Optics. 
  //London: Edward Arnold (1904), 340 p.


  \bibitem{frenel}
  \textit{Hecht E.}, Optics. 
  //Addison Wesley, 1987. 457 p.


  \bibitem{lambert}
  \textit{Frank Pedrotti, Leno Pedrotti}, Introduction to Optics. 
  // Prentice Hall, 1993. ISBN 0135015456.  

  \bibitem{rendering-equ}
  \textit{J.\,Kajiya}, The rendering equation
  // ACM SIGGRAPH Computer Graphics, 1986. 20. N 4. P. 143-150

  \bibitem{history}
  \textit{Андрей Лебедев}, История развития алгоритмов глобального освещения
  // Компьютерная графика и мультимедиа, 2011
  [\href{http://cgm.computergraphics.ru/issues/issue19/globalillum}
  {html}]

  \bibitem{ray-tracing}
  \textit{Cook R., Porter T., Carpenter L.}, Distributed ray tracing
  // SIGGRAPH Comput. Graph, 1984. 18. N 3. P. 137-145.

  \bibitem{geometry}
  \textit{Moller T., Trumbore B.}, Fast, Minimum Storage Ray-Triangle Intersection 
  // J. Graphics Tools, 1997, v.2(1), p.21 -- 28.

  \bibitem{brezenhem}
  \textit{Роджерс Д.}, Алгоритмические основы машинной графики
  //Мир, 1989. — С. 512.

  \bibitem{3dda}, Анализ алгоритмов трассировки лучей для реалистичной визуализации трехмерных сцен и способов уменьшения их вычислительной сложности. 

  \textit{И.А. Запорожченко, М.А. Григорьев, С.А. Зори}, 
  //Цифровая обработка сигналов и изображений. – с. 353.
\end{thebibliography}
