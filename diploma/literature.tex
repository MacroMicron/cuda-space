\newpage
\begin{thebibliography}{99}

  \bibitem{sazonov}
  \textit{С.Б.\,Медведев, В.В.\,Сазонов, Х.У.\,Сайгираев}, Моделирование зон неустойчивой работы радиотехнической измерительной системы с активным ответом во время сближения и стыковки космических кораблей с Международной Космической Станцией
  // Математическое моделирование, 24(2):151–160, 2012.
  [\href{http://istina.imec.msu.ru/publications/article/533721/}{html}]

  \bibitem{soyz}
  Космический корабль <<Союз>>
  [\href{http://ru.wikipedia.org/wiki/%D0%A1%D0%BE%D1%8E%D0%B7_(%D0%BA%D0%BE%D1%81%D0%BC%D0%B8%D1%87%D0%B5%D1%81%D0%BA%D0%B8%D0%B9_%D0%BA%D0%BE%D1%80%D0%B0%D0%B1%D0%BB%D1%8C)}{html}]

  \bibitem{boreskov1}
  \textit{А.В.\,Боресков, А.А.\,Харламов}, Основы работы с технологией CUDA
  // ДМК Пресс, 2010

  \bibitem{berezin}
  \textit{C.Б.\,Березин, В.М.\,Пасконов, Н.А.\,Сахарных}, Моделирование трехмерных течений методом расщепления с использованием параллельной архитектуры ГПУ
 // Вычислительные методы и программирование, 13: 75-81, 2012.
  [\href{http://num-meth.srcc.msu.ru/zhurnal/tom_2012/pdf/v13r210.pdf}{pdf}]

  \bibitem{cynami}
  \textit{М.А.\,Курако}, Оптимизация производительности вычислений для моделирования цунами на параллельных архитекутурах
  // Молодёжь и наука: Сборник материалов VII Всероссийской научно-технической конференции студентов, аспирантов и молодых учёных, посвященной 50-летию первого полета человека в космос
  [\href{http://elib.sfu-kras.ru/bitstream/2311/5869/1/s3_029.pdf}{pdf}]
  
  \bibitem{neuron}
  \textit{В.В.\,Парубец, О.Г.\,Берестнева, Д.В.\,Девятых}, Применение технологии CUDA для ускорения вычислений в нейронных сетях
  // Известия Томского политехнического университета, том 320, выпуск 5 (2012)

  \bibitem{radiolocation}
  \textit{П.В.\,Маковецкий, В.Г.\,Васильев}, Отражение радиолокационных сигналов -- лекции
  // Ленинградский институт авиационного приборостроения, 1975

  \bibitem{gugens}
  Принцип Гюйгенса-Френеля
  [\href{http://ru.wikipedia.org/wiki/%D0%9F%D1%80%D0%B8%D0%BD%D1%86%D0%B8%D0%BF_%D0%93%D1%8E%D0%B9%D0%B3%D0%B5%D0%BD%D1%81%D0%B0_%E2%80%94_%D0%A4%D1%80%D0%B5%D0%BD%D0%B5%D0%BB%D1%8F}
  {html}]

  \bibitem{ferma}
  Принцип Ферма
  [\href{http://ru.wikipedia.org/wiki/%D0%9F%D1%80%D0%B8%D0%BD%D1%86%D0%B8%D0%BF_%D0%A4%D0%B5%D1%80%D0%BC%D0%B0}
  {html}]

  \bibitem{frenel}
  Законы Френеля
  [\href{http://ru.wikipedia.org/wiki/%D0%A4%D0%BE%D1%80%D0%BC%D1%83%D0%BB%D1%8B_%D0%A4%D1%80%D0%B5%D0%BD%D0%B5%D0%BB%D1%8F}
  {html}]

  \bibitem{lambert}
  Законы Ламберта
  [\href{http://ru.wikipedia.org/wiki/%D0%97%D0%B0%D0%BA%D0%BE%D0%BD_%D0%9B%D0%B0%D0%BC%D0%B1%D0%B5%D1%80%D1%82%D0%B0}
  {html}]

  \bibitem{rendering-equ}
  \textit{J.\,Kajiya}, The rendering equation
  // ACM SIGGRAPH Computer Graphics, 1986. 20. N 4. P. 143-150

  \bibitem{history}
  \textit{Андрей Лебедев}, История развития алгоритмов глобального освещения
  // Компьютерная графика и мультимедиа, 2011
  [\href{http://cgm.computergraphics.ru/issues/issue19/globalillum}
  {html}]

  \bibitem{ray-tracing}
  \textit{Cook R., Porter T., Carpenter L.}, Distributed ray tracing
  // SIGGRAPH Comput. Graph, 1984. 18. N 3. P. 137-145.

\end{thebibliography}
