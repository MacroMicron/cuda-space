\newpage
\section*{Выводы} 

По результатам работы были достигнуты следующие цели:

1. написано программное обеспечение, помогающее в анализировании областей влияния зон первичного и вторичного отражения;

2. написанный параллельный вариант программы, позволяет рассчитывать сцену и создавать карты проекции за время, значительно меньшее, чем последовательный. 

Однако, расчет влияние зон вторичного отражения является трудоемким процессом, поэтому одним из путей улучшения решения задачи, приводимой в данной работе, является использование оптимальный алгоритмов, например, использование регулярной сетки. Все трехмерное пространство разбивается сеткой, а пересечение ищется только по тем кубикам, через которые прошел луч. Это совершается, например, алгоритмами Брезенхема \cite{brezenhem} или 3DDA-обходом (алгоритм Fujimoto \cite{3dda}).

CUDA -- достаточно сложный, но при этом гибкий инструмент. Добиться более высокой производительности приложения представляется возможным использованием многих низкоуровневых оптимизаций. Несмотря, на известные проблемы острой нехватки регистров графического процессора при трассировки лучей \cite{boreskov1} определенное ускорение путем оптимизации типа использованной памяти получить можно.
