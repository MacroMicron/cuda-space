\section*{Обзор методов решения уравнения рендеринга} 


------------------Способы решения уравнения рендеринга------------------------------
<<<решить вопрос L - сила излучения или энергетическая яркость, разница в размерности m в квадрате 2>>>


Алгоритм "бросания лучей" ray-casting.

Фактически, данный метод учитывает только первично-освещенные точки поверхности. Из точки выпускается луч по направлению к источнику. Если этот луч имеет пересечения с каким-либо иным фрагментом поверхности, то значит, что точка, из которой был выпущен луч, находится в тени. В ином случае, рассчитываем долю энергии получаемую точкой. [*история рендеринга?]

<<красивая картинка>>


Алгоритм visibility ray-tracing и трассировка фотонов.

Предыдущий алгоритм учитывал только первично-освещенные точки. Повысив количество 



Метод Монте-Карло (на базе трассировки лучей)

Все пространство разбивается равномерной сеткой. Количество узлов в трехмерном пространстве при этом равно $N^3$. Узлы, по которым ведется интегрирование, выбираются случайно. Соответственно, при увеличении числа узлов интеграл, вычисленный методом Монте-Карло, приближается к точному значению. Но при недостаточном количестве узлов получаем шумы. 
