\section*{Реализация}

1. ------------Трассировка лучей --------------

Для определения видимости полигона $ \alpha $ из точки $ L $, рассматривается отрезок $ LP $, где $ P $ -- центр полигона. Находятся все пересечения отрезка с полигонами модели (разумеется, многоугольник $ \alpha $ не подлежит рассмотрению). Если таких граней не нашлось, то грань $ \alpha $ считается видимой из точки $ L $, иначе -- невидимой. 

Задача определения пересечения отрезка и многоугольника разбивается на следующие этапы.

1. Определить точку пересечения $ M $ плоскости многоугольника и прямой, содержащей отрезок.

2. Находится ли точка пересечения $ M $ внутри многоугольника.

3. Находится ли точка пересечения $ M $ внутри отрезка.

Несмотря на то, что для достаточно детализированной модели с количеством полигонов в несколько тысяч, данный процесс достаточно трудоемок, в работе программе исходной статьи никакие способы оптимизации не использовались. Способов оптимизации действительно много и сравнение самых популярных можно прочитать в [*].

