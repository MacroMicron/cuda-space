\section*{Математическая постановка задачи}



3. ------------ Что делать с физикой отражения ------------------------------------
<<расчет энергии? нечто вроде того, что у Сазонова в работе? И то как прийти из физики к основному уравнению рендеринга>>


--------Математическая модель-----------------------

<<<излучает на 360 - супер антенная решетка>>>

Предположим, у нас есть ненаправленный антенна, представляемая точечным источником (расстояние до площадки значительно больше размеров антенны) с излучающей мощностью $P_0$. Тогда вся энергия распределяется равномерно по всей поверхности некоторой сферы радиуса $R$, площадь которой равна $ 4 \pi R^2 $. Таким образом, источник электромагнитного излучения создает плотность потока мощности
\begin{gather}
   J = \frac{P_0}{4 \pi R^2}
\end{gather}
Пусть также у нас имеется некоторая поверхность, расположенная в зоне прямой видимости источника излучения.
Тогда фрагмент поверхности $d\sigma$ получает в секунду энергию мощности от источника излучения

\begin{gather}
  P_in = J  d\sigma
\end{gather}

Тогда каждая точка поверхности $ M(x, y, z) $ получит энергию - ротор?
И поле J - векторное? Раз есть произведение на нормаль.

\begin{gather}
   P_in = J \cdot S \cdot cos \phi
\end{gather}


Часть энергии попадаемое в точку поверхности поглощаеся, часть отражается. Способ отражения от поверхности зададим в общем случае функцией 

Точка поверхности, с падаемым на неё излучением от антенны, становится источником вторичного излучения мощностью

Источники вторичного излучения (первично-освещаемые точки поверхности) светят, попадая на вторично-освещаемые точки поверхности, становясь при этом источниками третичного излучения. И так далее, процесс повторяется до бесконечности. 

В конечном итоге освещенность точек поверхности распределяется в соответствии с уравнением
\begin{gather}
   УРАВНЕНИЕ РЕНДЕРИНГА
\end{gather}

Данное уравнение называется уравнением рендеринга [*]. Существует несколько подходов к его решению.


------------------Способы решения уравнения рендеринга------------------------------
<<<решить вопрос L - сила излучения или энергетическая яркость, разница в размерности m в квадрате 2>>>

